\documentclass[]{article}
\usepackage{lmodern}
\usepackage{amssymb,amsmath}
\usepackage{ifxetex,ifluatex}
\usepackage{fixltx2e} % provides \textsubscript
\ifnum 0\ifxetex 1\fi\ifluatex 1\fi=0 % if pdftex
  \usepackage[T1]{fontenc}
  \usepackage[utf8]{inputenc}
\else % if luatex or xelatex
  \ifxetex
    \usepackage{mathspec}
  \else
    \usepackage{fontspec}
  \fi
  \defaultfontfeatures{Ligatures=TeX,Scale=MatchLowercase}
\fi
% use upquote if available, for straight quotes in verbatim environments
\IfFileExists{upquote.sty}{\usepackage{upquote}}{}
% use microtype if available
\IfFileExists{microtype.sty}{%
\usepackage{microtype}
\UseMicrotypeSet[protrusion]{basicmath} % disable protrusion for tt fonts
}{}
\usepackage[margin=1in]{geometry}
\usepackage{hyperref}
\hypersetup{unicode=true,
            pdftitle={Charactersing effect of anaemia on mortality in severe malaria},
            pdfborder={0 0 0},
            breaklinks=true}
\urlstyle{same}  % don't use monospace font for urls
\usepackage{color}
\usepackage{fancyvrb}
\newcommand{\VerbBar}{|}
\newcommand{\VERB}{\Verb[commandchars=\\\{\}]}
\DefineVerbatimEnvironment{Highlighting}{Verbatim}{commandchars=\\\{\}}
% Add ',fontsize=\small' for more characters per line
\usepackage{framed}
\definecolor{shadecolor}{RGB}{248,248,248}
\newenvironment{Shaded}{\begin{snugshade}}{\end{snugshade}}
\newcommand{\KeywordTok}[1]{\textcolor[rgb]{0.13,0.29,0.53}{\textbf{#1}}}
\newcommand{\DataTypeTok}[1]{\textcolor[rgb]{0.13,0.29,0.53}{#1}}
\newcommand{\DecValTok}[1]{\textcolor[rgb]{0.00,0.00,0.81}{#1}}
\newcommand{\BaseNTok}[1]{\textcolor[rgb]{0.00,0.00,0.81}{#1}}
\newcommand{\FloatTok}[1]{\textcolor[rgb]{0.00,0.00,0.81}{#1}}
\newcommand{\ConstantTok}[1]{\textcolor[rgb]{0.00,0.00,0.00}{#1}}
\newcommand{\CharTok}[1]{\textcolor[rgb]{0.31,0.60,0.02}{#1}}
\newcommand{\SpecialCharTok}[1]{\textcolor[rgb]{0.00,0.00,0.00}{#1}}
\newcommand{\StringTok}[1]{\textcolor[rgb]{0.31,0.60,0.02}{#1}}
\newcommand{\VerbatimStringTok}[1]{\textcolor[rgb]{0.31,0.60,0.02}{#1}}
\newcommand{\SpecialStringTok}[1]{\textcolor[rgb]{0.31,0.60,0.02}{#1}}
\newcommand{\ImportTok}[1]{#1}
\newcommand{\CommentTok}[1]{\textcolor[rgb]{0.56,0.35,0.01}{\textit{#1}}}
\newcommand{\DocumentationTok}[1]{\textcolor[rgb]{0.56,0.35,0.01}{\textbf{\textit{#1}}}}
\newcommand{\AnnotationTok}[1]{\textcolor[rgb]{0.56,0.35,0.01}{\textbf{\textit{#1}}}}
\newcommand{\CommentVarTok}[1]{\textcolor[rgb]{0.56,0.35,0.01}{\textbf{\textit{#1}}}}
\newcommand{\OtherTok}[1]{\textcolor[rgb]{0.56,0.35,0.01}{#1}}
\newcommand{\FunctionTok}[1]{\textcolor[rgb]{0.00,0.00,0.00}{#1}}
\newcommand{\VariableTok}[1]{\textcolor[rgb]{0.00,0.00,0.00}{#1}}
\newcommand{\ControlFlowTok}[1]{\textcolor[rgb]{0.13,0.29,0.53}{\textbf{#1}}}
\newcommand{\OperatorTok}[1]{\textcolor[rgb]{0.81,0.36,0.00}{\textbf{#1}}}
\newcommand{\BuiltInTok}[1]{#1}
\newcommand{\ExtensionTok}[1]{#1}
\newcommand{\PreprocessorTok}[1]{\textcolor[rgb]{0.56,0.35,0.01}{\textit{#1}}}
\newcommand{\AttributeTok}[1]{\textcolor[rgb]{0.77,0.63,0.00}{#1}}
\newcommand{\RegionMarkerTok}[1]{#1}
\newcommand{\InformationTok}[1]{\textcolor[rgb]{0.56,0.35,0.01}{\textbf{\textit{#1}}}}
\newcommand{\WarningTok}[1]{\textcolor[rgb]{0.56,0.35,0.01}{\textbf{\textit{#1}}}}
\newcommand{\AlertTok}[1]{\textcolor[rgb]{0.94,0.16,0.16}{#1}}
\newcommand{\ErrorTok}[1]{\textcolor[rgb]{0.64,0.00,0.00}{\textbf{#1}}}
\newcommand{\NormalTok}[1]{#1}
\usepackage{graphicx,grffile}
\makeatletter
\def\maxwidth{\ifdim\Gin@nat@width>\linewidth\linewidth\else\Gin@nat@width\fi}
\def\maxheight{\ifdim\Gin@nat@height>\textheight\textheight\else\Gin@nat@height\fi}
\makeatother
% Scale images if necessary, so that they will not overflow the page
% margins by default, and it is still possible to overwrite the defaults
% using explicit options in \includegraphics[width, height, ...]{}
\setkeys{Gin}{width=\maxwidth,height=\maxheight,keepaspectratio}
\IfFileExists{parskip.sty}{%
\usepackage{parskip}
}{% else
\setlength{\parindent}{0pt}
\setlength{\parskip}{6pt plus 2pt minus 1pt}
}
\setlength{\emergencystretch}{3em}  % prevent overfull lines
\providecommand{\tightlist}{%
  \setlength{\itemsep}{0pt}\setlength{\parskip}{0pt}}
\setcounter{secnumdepth}{0}
% Redefines (sub)paragraphs to behave more like sections
\ifx\paragraph\undefined\else
\let\oldparagraph\paragraph
\renewcommand{\paragraph}[1]{\oldparagraph{#1}\mbox{}}
\fi
\ifx\subparagraph\undefined\else
\let\oldsubparagraph\subparagraph
\renewcommand{\subparagraph}[1]{\oldsubparagraph{#1}\mbox{}}
\fi

%%% Use protect on footnotes to avoid problems with footnotes in titles
\let\rmarkdownfootnote\footnote%
\def\footnote{\protect\rmarkdownfootnote}

%%% Change title format to be more compact
\usepackage{titling}

% Create subtitle command for use in maketitle
\newcommand{\subtitle}[1]{
  \posttitle{
    \begin{center}\large#1\end{center}
    }
}

\setlength{\droptitle}{-2em}
  \title{Charactersing effect of anaemia on mortality in severe malaria}
  \pretitle{\vspace{\droptitle}\centering\huge}
  \posttitle{\par}
  \author{}
  \preauthor{}\postauthor{}
  \date{}
  \predate{}\postdate{}


\begin{document}
\maketitle

{
\setcounter{tocdepth}{2}
\tableofcontents
}
\section{Background}\label{background}

This looks at the severe malaria legacy dataset from MORU

The contributions of the different studies:

\begin{Shaded}
\begin{Highlighting}[]
\CommentTok{# Whole dataset}
\KeywordTok{table}\NormalTok{(Leg_data}\OperatorTok{$}\NormalTok{studyID)}
\end{Highlighting}
\end{Shaded}

\begin{verbatim}
## 
##          AAV           AQ     AQGambia      AQUAMAT Core Malaria 
##          370          560          579         5494         1121 
##    SEAQUAMAT 
##         1461
\end{verbatim}

\begin{Shaded}
\begin{Highlighting}[]
\CommentTok{# in the complete dataset (all variables recorded)}
\KeywordTok{table}\NormalTok{(Complete_Leg_data}\OperatorTok{$}\NormalTok{studyID)}
\end{Highlighting}
\end{Shaded}

\begin{verbatim}
## 
##          AAV           AQ     AQGambia      AQUAMAT Core Malaria 
##            0            0            0         3779          359 
##    SEAQUAMAT 
##         1090
\end{verbatim}

\section{Exploratory analysis}\label{exploratory-analysis}

Let's look at the key predictive variables. We use a random effects term
to model differences between studies.
\includegraphics{LegacyAnalysis_files/figure-latex/ExploratoryPlots-1.pdf}

\section{Predictive value of anaemia on death adjusting for
confounders}\label{predictive-value-of-anaemia-on-death-adjusting-for-confounders}

Before fitting the more complex GAM models we explore the standard glm
(logistic regression) models.

\begin{Shaded}
\begin{Highlighting}[]
\NormalTok{mod_full =}\StringTok{ }\KeywordTok{glmer}\NormalTok{(outcome }\OperatorTok{~}\StringTok{ }\NormalTok{HCT }\OperatorTok{+}\StringTok{ }\NormalTok{LPAR }\OperatorTok{+}\StringTok{ }\NormalTok{AgeInYear }\OperatorTok{+}\StringTok{ }\NormalTok{BUN }\OperatorTok{+}\StringTok{ }\NormalTok{BD }\OperatorTok{+}\StringTok{ }\NormalTok{drug }\OperatorTok{+}\StringTok{ }\NormalTok{(}\DecValTok{1} \OperatorTok{|}\StringTok{ }\NormalTok{studyID),}
               \DataTypeTok{data=}\NormalTok{Complete_Leg_data, }\DataTypeTok{family=}\NormalTok{binomial)}
\end{Highlighting}
\end{Shaded}

\begin{verbatim}
## Warning in checkConv(attr(opt, "derivs"), opt$par, ctrl = control
## $checkConv, : unable to evaluate scaled gradient
\end{verbatim}

\begin{verbatim}
## Warning in checkConv(attr(opt, "derivs"), opt$par, ctrl = control
## $checkConv, : Model failed to converge: degenerate Hessian with 1 negative
## eigenvalues
\end{verbatim}

\begin{Shaded}
\begin{Highlighting}[]
\KeywordTok{summary}\NormalTok{(mod_full)}
\end{Highlighting}
\end{Shaded}

\begin{verbatim}
## Warning in vcov.merMod(object, use.hessian = use.hessian): variance-covariance matrix computed from finite-difference Hessian is
## not positive definite or contains NA values: falling back to var-cov estimated from RX
\end{verbatim}

\begin{verbatim}
## Warning in vcov.merMod(object, correlation = correlation, sigm = sig): variance-covariance matrix computed from finite-difference Hessian is
## not positive definite or contains NA values: falling back to var-cov estimated from RX
\end{verbatim}

\begin{verbatim}
## Generalized linear mixed model fit by maximum likelihood (Laplace
##   Approximation) [glmerMod]
##  Family: binomial  ( logit )
## Formula: 
## outcome ~ HCT + LPAR + AgeInYear + BUN + BD + drug + (1 | studyID)
##    Data: Complete_Leg_data
## 
##      AIC      BIC   logLik deviance df.resid 
##   3199.6   3278.3  -1587.8   3175.6     5216 
## 
## Scaled residuals: 
##     Min      1Q  Median      3Q     Max 
## -2.7825 -0.3409 -0.2313 -0.1633 10.1058 
## 
## Random effects:
##  Groups  Name        Variance Std.Dev.
##  studyID (Intercept) 0.07967  0.2823  
## Number of obs: 5228, groups:  studyID, 3
## 
## Fixed effects:
##                    Estimate Std. Error z value Pr(>|z|)    
## (Intercept)      -1.850e+01  5.664e+02  -0.033  0.97394    
## HCT               1.960e-02  5.471e-03   3.583  0.00034 ***
## LPAR              1.855e-02  6.719e-02   0.276  0.78247    
## AgeInYear         2.186e-02  4.343e-03   5.033 4.83e-07 ***
## BUN               1.166e-02  1.694e-03   6.882 5.89e-12 ***
## BD                1.361e-01  6.944e-03  19.601  < 2e-16 ***
## drugArtesunate    1.404e+01  5.664e+02   0.025  0.98022    
## drugChloroquine   1.610e+01  5.664e+02   0.028  0.97733    
## drugLumefantrine -1.034e+00  4.187e+03   0.000  0.99980    
## drugNAC          -5.245e+00  8.479e+03  -0.001  0.99951    
## drugQuinine       1.438e+01  5.664e+02   0.025  0.97975    
## ---
## Signif. codes:  0 '***' 0.001 '**' 0.01 '*' 0.05 '.' 0.1 ' ' 1
## 
## Correlation of Fixed Effects:
##             (Intr) HCT    LPAR   AgInYr BUN    BD     drgArt drgChl drgLmf
## HCT          0.000                                                        
## LPAR        -0.001 -0.172                                                 
## AgeInYear    0.000 -0.183  0.030                                          
## BUN          0.000  0.064 -0.050 -0.189                                   
## BD           0.000  0.263 -0.135  0.120 -0.263                            
## drugArtesnt -1.000  0.000  0.000  0.000  0.000  0.000                     
## drugChlorqn -1.000  0.000  0.000  0.000  0.000  0.000  1.000              
## drugLmfntrn -0.135  0.000  0.000  0.000  0.000  0.000  0.135  0.135       
## drugNAC     -0.067  0.000  0.000  0.000  0.000  0.000  0.067  0.067  0.009
## drugQuinine -1.000  0.000  0.000  0.000  0.000  0.000  1.000  1.000  0.135
##             drgNAC
## HCT               
## LPAR              
## AgeInYear         
## BUN               
## BD                
## drugArtesnt       
## drugChlorqn       
## drugLmfntrn       
## drugNAC           
## drugQuinine  0.067
## convergence code: 0
## unable to evaluate scaled gradient
## Model failed to converge: degenerate  Hessian with 1 negative eigenvalues
\end{verbatim}

Now let's make counterfactual predictions of anaemia on death for the
patients in the database. The way to interpret this `counterfactual'
plot is as follows: suppose that every individual in the dataset was
assigned (as in a intervention) a specific haematocrit \(X\), what would
the resulting per patient probability of death be. Here we summarise
these probabilities by the predicted mean probability of death and 80\%
predictive intervals.

\begin{Shaded}
\begin{Highlighting}[]
\NormalTok{overall_mortality =}\StringTok{ }\DecValTok{100}\OperatorTok{*}\KeywordTok{mean}\NormalTok{(Complete_Leg_data}\OperatorTok{$}\NormalTok{outcome)}
\KeywordTok{par}\NormalTok{(}\DataTypeTok{las=}\DecValTok{1}\NormalTok{, }\DataTypeTok{bty=}\StringTok{'n'}\NormalTok{)}
\NormalTok{x_hcts =}\StringTok{ }\KeywordTok{seq}\NormalTok{(}\DecValTok{4}\NormalTok{,}\DecValTok{45}\NormalTok{, }\DataTypeTok{by=}\NormalTok{.}\DecValTok{5}\NormalTok{)}
\NormalTok{probs =}\StringTok{ }\KeywordTok{array}\NormalTok{(}\DataTypeTok{dim =} \KeywordTok{c}\NormalTok{(}\DecValTok{3}\NormalTok{, }\KeywordTok{length}\NormalTok{(x_hcts)))}
\ControlFlowTok{for}\NormalTok{(i }\ControlFlowTok{in} \DecValTok{1}\OperatorTok{:}\KeywordTok{length}\NormalTok{(x_hcts))\{}
\NormalTok{  mydata =}\StringTok{ }\NormalTok{Complete_Leg_data}
\NormalTok{  mydata}\OperatorTok{$}\NormalTok{HCT=x_hcts[i]}
\NormalTok{  ys =}\StringTok{ }\DecValTok{100}\OperatorTok{*}\KeywordTok{predict}\NormalTok{(mod_full, }\DataTypeTok{newdata =}\NormalTok{ mydata, }\DataTypeTok{re.form=}\OtherTok{NA}\NormalTok{, }\DataTypeTok{type=}\StringTok{'response'}\NormalTok{)}
\NormalTok{  probs[}\DecValTok{2}\NormalTok{,i] =}\StringTok{ }\KeywordTok{mean}\NormalTok{(ys)}
\NormalTok{  probs[}\KeywordTok{c}\NormalTok{(}\DecValTok{1}\NormalTok{,}\DecValTok{3}\NormalTok{),i] =}\StringTok{ }\KeywordTok{quantile}\NormalTok{(ys, }\DataTypeTok{probs=}\KeywordTok{c}\NormalTok{(}\FloatTok{0.1}\NormalTok{,}\FloatTok{0.9}\NormalTok{))}
\NormalTok{\}}
\KeywordTok{plot}\NormalTok{(x_hcts,probs[}\DecValTok{2}\NormalTok{,], }\DataTypeTok{xlim=}\KeywordTok{c}\NormalTok{(}\DecValTok{4}\NormalTok{,}\DecValTok{45}\NormalTok{), }\DataTypeTok{ylab=}\StringTok{'Predicted probability of death'}\NormalTok{, }
     \DataTypeTok{xlab=}\StringTok{'Haematocrit (%)'}\NormalTok{, }\DataTypeTok{ylim=}\KeywordTok{c}\NormalTok{(}\DecValTok{0}\NormalTok{,}\DecValTok{50}\NormalTok{), }\DataTypeTok{lty=}\DecValTok{1}\NormalTok{, }\DataTypeTok{lwd=}\DecValTok{3}\NormalTok{, }\DataTypeTok{type=}\StringTok{'l'}\NormalTok{)}
\KeywordTok{lines}\NormalTok{(x_hcts, probs[}\DecValTok{1}\NormalTok{,], }\DataTypeTok{lty=}\DecValTok{2}\NormalTok{, }\DataTypeTok{lwd=}\DecValTok{2}\NormalTok{)}
\KeywordTok{lines}\NormalTok{(x_hcts, probs[}\DecValTok{3}\NormalTok{,], }\DataTypeTok{lty=}\DecValTok{2}\NormalTok{, }\DataTypeTok{lwd=}\DecValTok{2}\NormalTok{)}
\KeywordTok{abline}\NormalTok{(}\DataTypeTok{h=}\NormalTok{overall_mortality, }\DataTypeTok{lwd=}\DecValTok{3}\NormalTok{, }\DataTypeTok{col=}\StringTok{'blue'}\NormalTok{,}\DataTypeTok{lty=}\DecValTok{2}\NormalTok{)}
\KeywordTok{legend}\NormalTok{(}\StringTok{'topleft'}\NormalTok{, }\DataTypeTok{col=}\KeywordTok{c}\NormalTok{(}\StringTok{'black'}\NormalTok{,}\StringTok{'black'}\NormalTok{,}\StringTok{'blue'}\NormalTok{), }\DataTypeTok{lwd=}\DecValTok{3}\NormalTok{, }\DataTypeTok{lty=}\KeywordTok{c}\NormalTok{(}\DecValTok{1}\NormalTok{,}\DecValTok{2}\NormalTok{,}\DecValTok{2}\NormalTok{),}
       \DataTypeTok{legend =} \KeywordTok{c}\NormalTok{(}\StringTok{'Mean predicted mortality'}\NormalTok{, }\StringTok{'80% predicted interval'}\NormalTok{,}\StringTok{'Observed mortality'}\NormalTok{))}
\end{Highlighting}
\end{Shaded}

\includegraphics{LegacyAnalysis_files/figure-latex/counterfactualMortality-1.pdf}


\end{document}
